\documentclass[english,notitlepage,reprint,nofootinbib]{revtex4-2}

\usepackage[noend]{algpseudocode}
\usepackage[margin=2cm]{geometry}
\usepackage{physics, amssymb}
\usepackage[utf8]{inputenc}
\usepackage{subcaption}
\usepackage{algorithm}
\usepackage{graphicx}
\usepackage{hyperref}
\usepackage{listings}
\usepackage{minted}
\usepackage{xcolor}
\usepackage{float}
\usepackage{tikz}
\usepackage{svg}

\usetikzlibrary{quantikz}

\hypersetup{
    colorlinks,
    linkcolor={red!50!black},
    citecolor={blue!50!black},
    urlcolor={blue!80!black}
}

\begin{document}

\title{The Ising model}
\author{Emilie Dørum}
\date{\today}
\noaffiliation

\begin{abstract}
\end{abstract}
\maketitle

\newcommand{\dunder}[1] {
	\underline{\underline{#1}} 
}

\newcommand{\mat}[1] {
	\begin{bmatrix}#1\end{bmatrix}
}

\newcommand{\bs}[1] {
	\boldsymbol{#1}
}

\newcommand{\bhat}[1] {
	\hat{\boldsymbol{#1}}
}

\newcommand{\vary}[1] {
	\text{Var}\left[#1\right]
}

\newcommand{\expy}[1] {
	\mathbb{E}\left[#1\right]
}

\newcommand{\avg}[1] {
	\left<#1\right>
}

\section{Introduction}

\section{Methods}

\subsection{Implementation}

\section{Results}

\section{Discussion}

\section{Conclusion}

\onecolumngrid

\section*{References}
\begin{thebibliography}{}
\end{thebibliography}

\clearpage

\twocolumngrid

\appendix

\section{Code}

The code used to make the figures in this paper can be found on GitHub at \\
\href{https://github.com/Emilinya/FYS4150/tree/main/project3}
{https://github.com/Emilinya/FYS4150/tree/main/project3}

\section{Analytical solution}\label{anasol}
Assume a $2 \times 2$ lattice with periodic boundary conditions. This lattice has 16 possible spin states, and looks like
\begin{center}
    \begin{tabular}{c|c}
        $s_1$ & $s_2$ \\
        \hline
        $s_3$ & $s_4$ \\
    \end{tabular}
\end{center} 

We can compute the sum over all neighboring pairs of spins without double-counting
$$
    \sum_{\avg{kl}}^4 s_ks_l = s_1s_2 + s_1s_3 + s_2s_4 + s_3s_4 = (s_1 + s_4)(s_2 + s_3)
$$

We first want to summarize all possible states of the system:
\begin{center}
    \begin{tabular}{|c|c|c|c|}
        \hline
        \# positive spins & E & M & Degeneracy \\
        \hline
        4 & $-4J$ & 4 & 1 \\
        3 & 0 & 2 & 4 \\
        2 & 0 & 0 & 4 \\
        2 & $4J$ & 0 & 2 \\
        1 & 0 & -2 & 4 \\
        0 & $-4J$ & -4 & 1 \\
        \hline
    \end{tabular}
\end{center}

We then want to calculate important quantities:
\begin{align*}
    Z &= \sum_s e^{-\beta E(s)}  \\
    &= 2e^{4\beta J} + 2e^{-4\beta J} + 12 \\
    &= 4(\cosh(4\beta J) + 3) \\
    \avg{E} &= \frac{2e^{4\beta J}(-4J) + 2e^{-4\beta J}(4J)}{4(\cosh(4\beta J) + 3)} \\
    &= J\frac{2\qty(e^{-4\beta J} - e^{4\beta J})}{\cosh(4\beta J) + 3} \\
    &= J\frac{4\sinh(-4\beta J)}{\cosh(4\beta J) + 3} \\
    \implies \avg{\epsilon} &= \frac{J}{N}\frac{4\sinh(-4\beta J)}{\cosh(4\beta J) + 3} \\
    \avg{E^2} &= \frac{2e^{4\beta J}(-4J)^2 + 2e^{-4\beta J}(4J)^2}{4(\cosh(4\beta J) + 3)} \\
    &= J^2\frac{8\qty(e^{4\beta J} + e^{-4\beta J})}{\cosh(4\beta J) + 3} \\
    &= J^2\frac{16\cosh(4\beta J)}{\cosh(4\beta J) + 3} \\
    \implies \avg{\epsilon^2} &= \frac{J^2}{N^2}\frac{16\cosh(4\beta J)}{\cosh(4\beta J) + 3} \\
    \avg{|M|} &= \frac{e^{4\beta J}4 + 4\cdot 2 + 4 |-2| + e^{4\beta J}|-4|}{4(\cosh(4\beta J) + 3)} \\
    &= 2\frac{e^{4\beta J} + 2}{\cosh(4\beta J) + 3} \\
    \implies \avg{|m|} &= \frac{2}{N}\frac{e^{4\beta J} + 2}{\cosh(4\beta J) + 3} \\
    \avg{M^2} &= \frac{e^{4\beta J}4^2 + 4\cdot 2^2 + 4 (-2)^2 + e^{4\beta J}(-4)^2}{4(\cosh(4\beta J) + 3)} \\
    &= 8\frac{e^{4\beta J} + 1}{\cosh(4\beta J) + 3} \\
    \implies \avg{m^2} &= \frac{8}{N^2}\frac{e^{4\beta J} + 1}{\cosh(4\beta J) + 3} \\
\end{align*}
\begin{align*}
    C_V &= \frac{1}{N} \frac{1}{k_B T^2}\qty(\avg{E^2} - \avg{E}^2) \\
    &= \frac{1}{N} \frac{1}{k_B T^2}\qty(J^2\frac{16\cosh(4\beta J)}{\cosh(4\beta J) + 3} - \qty(J\frac{4\sinh(-4\beta J)}{\cosh(4\beta J) + 3})^2) \\
    &= \frac{1}{N} \frac{16J^2}{k_B T^2}\frac{\cosh[2](4\beta J) + 3\cosh(4\beta J) - \sinh[2](4\beta J)}{(\cosh(4\beta J) + 3)^2} \\
    &= \frac{1}{N} \frac{16J^2}{k_B T^2}\frac{1 + 3 \cosh(4\beta J)}{(3 + \cosh(4\beta J))^2} \\
    \chi &= \frac{1}{N} \frac{1}{k_B T}(\avg{M^2} - \avg{|M|}^2) \\
    &= \frac{1}{N} \frac{1}{k_B T}\qty(8\frac{e^{4\beta J} + 1}{\cosh(4\beta J) + 3} - \qty(2\frac{e^{4\beta J} + 2}{\cosh(4\beta J) + 3})^2) \\
    &= \frac{1}{N} \frac{4}{k_B T}\qty(2\frac{e^{4\beta J} + 1}{\cosh(4\beta J) + 3} - \frac{\qty(e^{4\beta J} + 2)^2}{\qty(\cosh(4\beta J) + 3)^2}) \\
    &= \frac{1}{N} \frac{4}{k_B T} \frac{3e^{4\beta J} + e^{-4\beta J} + 2}{(\cosh(4\beta J) + 3)^2}
\end{align*}

In a 2D lattice of arbitrary size, the contribution of a spin $s_{i,j}$ to the total energy is 
$$e_{i,j} = -Js_{i,j}(s_{i+1,j} + s_{i-1,j} + s_{i,j+1} + s_{i,j-1})$$
Flipping the spin $s_{ij}$ is the same as multiplying $e_{i,j}$ by $-1$. This means that the spin flip results in an energy change equal to 
\begin{align*}
    \Delta E &= (-e_{i,j}) - e_{i,j} \\ 
    &= 2Js_{i,j}(s_{i+1,j} + s_{i-1,j} + s_{i,j+1} + s_{i,j-1})
\end{align*}
The value of $\Delta E$ then depends on the spin $s_{i,j}$, and the neighboring spins.  \\
If all the neighboring spins have the same sign, we get $\Delta E = 8J$ or $\Delta E = -8J$. \\
If one of the signs differ from the others, we get $\Delta E = 4J$ or $\Delta E = -4J$. \\
If two spins are positive and two are negative, we get $\Delta E = 0$.  \\
This means that $\Delta E$ only has 5 possible values,
$$
    \Delta E \in \qty{-8J, -4J, 0, 4J, 8J}
$$ 
We can use this result to avoid repeatedly calling the exponential function in our code by precomputing the Boltzmann factor $e^{-\beta \Delta E}$
for the five possible values, and then creating a map from energy change to Boltzmann factor. We can then look up values in that map when we want
to find the Boltzmann factor.

\end{document}
